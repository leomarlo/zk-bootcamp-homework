\documentclass[a4paper,10pt]{article}
\usepackage[utf8]{inputenc}
\usepackage{amsmath}
\usepackage{amsfonts}

%opening
\title{Answers to Homework 1}
\author{Leonhard Horstmeyer (Leo Marlo\#1048)}

\begin{document}

\maketitle

\section*{Question 1}
\paragraph{Odd Squares:}
I think I can actually prove this.
So we know that 
\begin{equation}
(a+b) \;\text{mod} \;N \equiv a \;\text{mod}\; N + b \;\text{mod} \;N 
\end{equation}
from modular arithmetics. In other words (writing out the definition of equivalence $\equiv$):
\begin{equation}(a+b) \;\text{mod}\; N = \left( a \;\text{mod}\; N + b \;\text{mod}\; N\right) \;\text{mod}\; N
\end{equation}
So let $x$ be some odd number, that is for any $x$ there exists a natural number $y\in \mathbb N$ such that $x=2y+1$.\\
We prove by induction. For $x=1$ the statement is obviously true, since $1^2 \;\text{mod}\; 8 = 1$. Suppose now that $x^2 \;\text{mod}\; 8 = c$, where $c=1$ in our case. Then consider the next odd number $x+2$:
\begin{align}
(x+2)^2 \; \text{mod} \; 8 
=& \left(x^2 + 4x + 4\right)\;\text{mod} \; 8 \nonumber\\
=& \left(x^2\;\text{mod} \; 8 + \left(4x\right)\text{mod} \; 8 + 4 \;\text{mod} \; 8 \right)\;\text{mod} \; 8
\nonumber\\
=& \left( c + \left(4x\right)\text{mod} \; 8  + 4 \right) \;\text{mod} \; 8 \nonumber\\
=& \left( c + \left(8 y + 4\right)\text{mod} \; 8  + 4 \right) \;\text{mod} \; 8 \nonumber\\
=& \left( c + \left((8 y) \text{mod} \; 8  + 4 \;\text{mod} \; 8 \right)\text{mod} \; 8  + 4 \right) \;\text{mod} \; 8  \nonumber\\
=& \left( c + 4 \;\text{mod} \; 8 + 4 \right) \;\text{mod} \; 8 \nonumber\\
=& \;(c+8)\text{mod} \; 8  =  c = 1
\end{align}

\paragraph{Even squares:}
For even squares the situation is the same, except that the result is always 4, i.e. $x^2 \;\text{mod} \; 8 = 4$. For $x=2$ the result is obviously true. Let's now assume it is true for some even $x$, then for $x+2$ we go through the same calculation as above, except that we take $c=4$.


\section*{Question 3}
\begin{enumerate}
 \item $\mathcal{O}(n)$ means that as we take the limit $n\to \infty$ the thing divided by $n$ converges to a non-zero and non-infinite number.
 \item  $\mathcal{O}(1)$ means that as we take the limit $n\to \infty$ the thing converges to a non-zero and non-infinite number.
 \item $\mathcal{O}(\log n)$ means that as we take the limit $n\to \infty$ the thing divided by $\log n$ converges to a non-zero and non-infinite number.
\end{enumerate}

Here  $\mathcal{O}(n)$ is the worst behaved, then comes $\mathcal{O}(\log n)$. The best is $\mathcal{O}(1)$.






\end{document}
