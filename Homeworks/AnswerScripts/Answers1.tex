\documentclass[a4paper,10pt]{article}
\usepackage[utf8]{inputenc}
\usepackage{amsmath}
\usepackage{amsfonts}

%opening
\title{Answers to Homework 1}
\author{Leonhard Horstmeyer (Leo Marlo\#1048)}

\begin{document}

\maketitle

\section*{Question 1}
Let $S=\mathbb Z_7$ and let $+$ and $\star$ be addition and multiplication respectively on $\mathbb Z_7$, then
\begin{itemize}
 \item[a)] $4+4=1$
 \item[b)] $3\star 5=1$
 \item[c)] The multiplicative inverse of $3$ is $5$ because $3\star 5$ yields the multiplicative unit element $1$. The additive inverse should be $4$, because $3+4$ yields the additive unit element $0$.
\end{itemize}

\section*{Question 2}
Yes we can cosider $(\mathbb Z_7, +)$ to be a group. It is closed, because the modular thingy makes the range of target values lie between $0$ and $6$. There is a unit element, namely $0$, for each element $a\in \mathbb Z_7$ there exists an element $b$ such that $a+b=0$, namely $b=7-a$. 

\section*{Question 3}
So $-13\;\text{mod}\;5 = (-13 + 15)\;\text{mod}\;5 = 2\;\text{mod}\;5= 2$.

\section*{Question 4}
According to Wolfram alpha 2 is a positive root, the other two roots are complex numbers with non zero imaginary part so that positivity does not have a meaning unless one refers to just the real part, in which case they are both negative.






\end{document}
